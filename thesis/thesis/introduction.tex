\chapter{Introduction}

In the field of computer graphics, one of the most common approaches to displaying 3D objects on a screen is by defining them as collections of flat shapes. The most widely used shape for this purpose is the triangle. A group of these triangles is called a mesh, which represents the object's surface. Although this technique has many advantages, it also has notable limitations. For instance, it is very challenging to model phenomena like clouds or fire (non physical objects) using meshes. Additionally, creating high-quality 3D meshes often requires skilled artists, it involves both a time-consuming and labor-intensive process.

Recent advancements in computer graphics have introduced a novel method for creating and visualizing 3D objects, called Gaussian splatting. Unlike traditional meshes, this technique represents objects using a collection of Gaussian points. A Gaussian point can be thought of as a 3D ellipsoid that fades out the further away from its center. This fade follows a Gaussian distribution based on the shape of an ellipsoid defined by the Gaussian point. Individually, these points may appear random, but when combined in sufficient numbers, they form structures resembling a cloud, capable of approximating the shape and appearance of any object and its environment. 

One significant advantage of Gaussian Splatting is how these points are generated. Instead of manually creating them, all Gaussian points are generated from a given input, such as a set of images or a video. From this data, a sparse point cloud is generated. Then, using this point cloud and the images, an optimization algorithm refines the representation. This algorithm combines gradient descent, point splitting, and merging to align Gaussian points with input images. Recent improvements in these optimization techniques have led to increasingly accurate and realistic results. The outcome is a highly realistic representation of the object, created with minimal manual effort, although it requires substantial GPU power to process.

However, Gaussian Splatting faces a critical usability challenge. Over the past decades, numerous tools and techniques have been developed for editing and manipulating meshes, including texturing, shape modification, animation, and physics simulations. Current tools are designed to work with meshes and are therefore incompatible with Gaussian splatting, which severely limits its adoption and practical application. Addressing this gap is the primary focus of this thesis. Rather than undertaking the enormous task of recreating all existing mesh-based editing techniques for Gaussian splats from scratch, this thesis proposes a solution that bridges the gap between the two approaches. The idea is to establish a connection between Gaussian splats and traditional meshes, enabling the reuse of existing mesh-editing tools for Gaussian-based representations.