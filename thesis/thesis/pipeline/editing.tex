\chapter{Editing Pipeline}

To enable intuitive editing of Gaussian splats, this thesis proposes a novel pipeline that integrates volumetric representations with traditional mesh-based workflows by constructing a dynamic, editable mesh as an intermediary structure. Unlike conventional workflows, in which meshes are the final rendered output, this approach uses the mesh solely as an invisible scaffold for indirect manipulation. Each Gaussian splat is spatially linked to the mesh's vertices, edges, or faces, allowing artists to control the Gaussian representation through familiar mesh-editing techniques. This method preserves the photorealism and automation of Gaussian splatting while maximizing compatibility with existing mesh-based tools, including deformations, texture mapping, and animations.

\section*{Core Requirements}
The proposed pipeline adheres to four critical principles:
\begin{itemize}
    \item \textbf{Generality}: Applicable to a wide range of pre-trained Gaussian splats without requiring access to their optimization data, training images, or camera parameters.
    \item \textbf{Invertibility}: Every transformation must be mathematically reversible to restore the original splat configuration when edits are reset.
    \item \textbf{Real-time Performance}: Edits should propagate with negligible latency to support interactive, artist-driven workflows.
    \item \textbf{Applicability}: Edits must be permanently baked into the Gaussian splats for use in downstream applications without dependency on the intermediary mesh.
\end{itemize}


The \textbf{generality} requirement prioritizes practical usability. In real-world scenarios, users often lack access to the original training data or the computational resources required to retrain Gaussian splats. By working exclusively with pre-optimized splats, the pipeline supports common use cases like editing downloaded 3D assets. However, this constraint presents a fundamental trade-off: custom properties (e.g., semantic labels, physical parameters) cannot be jointly optimized with the splat after the fact, limiting the range of information provided.

\section*{Pipeline Stages}
The editing workflow unfolds in three stages:
\begin{enumerate}
    \item \textbf{Mesh Extraction}: Derive a mesh from the Gaussian splats' implicit geometry, based on the distributions of each Gaussian point.
    \item \textbf{Linking}: Define bidirectional spatial-probabilistic mappings between Gaussian points and mesh anchor points (vertices, edges, faces).
    \item \textbf{Feature Propagation}: Translate mesh edits (e.g., vertex displacement, texture updates) into corresponding adjustments of the Gaussian splats' parameters (position, rotation, scale, opacity) via the established links.
\end{enumerate}

By decoupling editing from rendering, the pipeline enables artists to manipulate Gaussian splats using standard mesh tools (e.g., Blender, Maya) while preserving their volumetric fidelity. The following sections detail each stage: Section \ref{mesh-extraction} addresses mesh extraction, Section \ref{linking} formalizes the linking mechanism, and Section \ref{feature-propagation} defines the propagation framework for invertible edits in real-time.


