\chapter{Applications}

\label{applications}

The proposed pipeline fundamentally changes how Gaussian splats can be used in practice, transforming them from static, hard-to-edit representations into dynamic, artist-friendly assets. By linking Gaussian splats to an editable mesh scaffold, the pipeline unlocks a wide range of creative and technical possibilities that were previously inaccessible. This chapter explores the diverse applications enabled by this approach, showcasing how the pipeline bridges the gap between the realism of volumetric representations and the flexibility of traditional mesh-based workflows.

At its core, the pipeline allows artists and developers to manipulate Gaussian splats using familiar tools and techniques, such as vertex editing, rigging, texture painting, and physics simulations. This capability extends far beyond simple edits, enabling complex workflows like character animation, material design, and interactive environment creation. For instance, animators can rig a Gaussian-based character to a mesh skeleton, allowing for dynamic poses and expressions, while environment artists can paint textures directly onto splats, enhancing their visual fidelity. Additionally, the pipeline supports real-time interactions, such as collision detection and lighting adjustments, making Gaussian splats viable for use in games, simulations, and virtual production.

The following sections delve into specific application areas, organized into three categories: shape modification \ref{app::shape-modiers}, color editing \ref{app::color-editing}, and environment interaction \ref{app::env-interactions}. Each section highlights how the pipeline expands the utility of Gaussian Splatting, enabling new workflows and creative possibilities.

\section{Shape Modifiers}

\label{app::shape-modiers}

\input{thesis/applications/shape_modifiers/vertex_modifiers}
\input{thesis/applications/shape_modifiers/rigging}
\input{thesis/applications/shape_modifiers/animations}

\section{Color Modifiers}

\label{app::color-editing}

\input{thesis/applications/color_modifiers/color_modifiers}
\input{thesis/applications/color_modifiers/texture_painting}

\section{Environment Interaction}

\label{app::env-interactions}

\input{thesis/applications/environment_interaction/hitboxes}
\input{thesis/applications/environment_interaction/lighting}
\input{thesis/applications/environment_interaction/shadows}