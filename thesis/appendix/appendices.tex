\newcommand{\nocontentsline}[3]{}
\let\origcontentsline\addcontentsline
\newcommand\stoptoc{\let\addcontentsline\nocontentsline}
\newcommand\resumetoc{\let\addcontentsline\origcontentsline}

\chapter{Data Structure Representations}
\stoptoc

This appendix provides an overview of various mesh data structures commonly used in computational geometry and computer graphics. Each structure is presented with its corresponding code representation.

\section{Triangle Neighbor Structure}
\label{app:triangle-neighbor-structure}

The \textbf{Triangle Neighbor Structure} enhances mesh traversal by storing adjacency information between triangles. Each triangle maintains references to its three neighboring triangles and associated vertices, facilitating efficient local connectivity queries.

\begin{minted}{cpp}
struct Triangle {
    Edge nbr[3];
    Vertex vertex[3];
};

struct Edge {
    // The i-th edge of triangle t
    Triangle t;
    int i; // Index in {0,1,2}
    // In practice, t and i share 32 bits
};

struct Vertex {
    // Per-vertex data
    Edge e; // Any edge leaving the vertex
};
\end{minted}

\section{Winged-Edge Structure}
\label{app:winged-edge-structure}

The \textbf{Winged-Edge Structure} provides a comprehensive representation of mesh connectivity by storing references to the adjacent faces, vertices, and neighboring edges. This enables efficient traversal and modification of the mesh topology.

\begin{minted}{cpp}
struct Edge {
    Edge *hl, *hr, *tl, *tr; // Adjacent edges
    Vertex *h, *t;           // Head and tail vertices
    Face *l, *r;             // Left and right adjacent faces
};

struct Face {
    // Per-face data
    Edge *e; // Any adjacent edge
};

struct Vertex {
    // Per-vertex data
    Edge *e; // Any incident edge
};
\end{minted}

\section{Half-Edge Representation}
\label{app:half-edge-representation}

The \textbf{Half-Edge Representation} is a widely used data structure that splits each edge into two directed half-edges, allowing for efficient traversal and topological modifications. Each half-edge stores references to its opposite, next half-edge, and the associated vertex and face.

\begin{minted}{cpp}
struct HEdge {
    HEdge *pair, *next;  // Opposite and next half-edge
    Vertex *v;           // Originating vertex
    Face *f;             // Associated face
};

struct Face {
    // Per-face data
    HEdge *h; // Any adjacent half-edge
};

struct Vertex {
    // Per-vertex data
    HEdge *h; // Any incident half-edge
};
\end{minted}

\resumetoc